\cchapter{ЗАКЛЮЧЕНИЕ}                       % Заголовок
\addcontentsline{toc}{chapter}{ЗАКЛЮЧЕНИЕ}  % Добавляем его в оглавление

%Заключение должно содержать:
%\begin{itemize}
%  \item краткие выводы по результатам выполненной ВКР или отдельных её этапов;
%  \item оценку полноты решений поставленных задач;
%  \item разработку рекомендаций и исходных данных по конкретному использованию результатов ВКР;
%  \item результаты оценки технико-экономической эффективности внедрения (если имеет место);
%  \item результаты оценки научно-технического уровня выполненной ВКР в сравнении с достижениями в этой области.
%\end{itemize}

В рамках данной выпускной квалификационной работы была достигнута поставленная цель — разработан сервис, позволяющий создавать цифровых аватаров на основе пользовательских данных. В процессе реализации были решены следующие задачи: обеспечена возможность обучения аватаров на загружаемых аудио-, видео- и текстовых данных, реализована функция текстового взаимодействия с обученным аватаром, а также возможность озвучивания сообщений с использованием синтезированной речи.

До начала разработки была проведена аналитическая работа: изучены существующие решения на рынке, а также исследована патентная база, что позволило учесть текущие тренды и избежать дублирования. На основании анализа фреймворков и библиотек произведена архитектурная декомпозиция сервиса на независимые компоненты, что обеспечило гибкость и масштабируемость при разработке.

Результатом работы стал прототип полноценного сервиса с веб-интерфейсом, реализующий весь необходимый функционал.

В перспективе дальнейшего развития сервиса, при наличии более высоких вычислительных ресурсов, возможно использование более сложных языковых и акустических моделей, что позволит значительно повысить качество взаимодействия с цифровыми аватарами. Кроме того, увеличение вычислительных мощностей откроет возможность ускоренной обработки запросов и позволит рассматривать направление генерации видео-аватаров, что расширит функциональность системы.