\chapter{Постановка задачи} 
Основной целью разрабатываемого сервиса является предоставление пользователям удобного и понятного интерфейса для 
взаимодействия с цифровыми двойниками реальных личностей — интеллектуальными аватарами. Сервис должен позволять не только 
вести диалог с аватаром посредством текстового и аудио общения, но и иметь возможность полноценного обучения аватара с 
использованием широкого спектра материалов, таких как тексты, аудио и видео. В результате аватар должен максимально точно 
воспроизводить индивидуальные особенности манеры общения и эмоционального отклика выбранного человека.

Для достижения поставленной цели необходимо реализовать следующие технические требования к минимальной конфигурации сервиса:

\begin{itemize}
\item Разработка веб-интерфейса, предоставляющего пользователям возможность 
взаимодействовать с созданными аватарами, задавать им вопросы и получать текстовые и аудио 
ответы в реальном времени.
\item Реализация удобного и интуитивного интерфейса для процесса обучения аватаров, который 
позволит пользователям загружать материалы различных форматов (тексты, аудио и видеозаписи) 
для последующей обработки и обучения моделей.
\item Обеспечение стабильности и надежности процесса обучения языковых и звуковых моделей, 
который является наиболее ресурсоёмкой частью работы сервиса и требует эффективного 
управления вычислительными ресурсами.
\item Поддержка возможности интеграции дополнительных источников данных и сторонних 
сервисов, таких как социальные сети и мессенджеры, для получения дополнительных материалов, 
обогащающих процесс обучения аватаров.
\item Реализация механизма хранения и управления данными разного типа и формата, включая 
метаданные аватаров, пользовательские данные, текстовые и аудио материалы для обучения, а 
также системные логи.
\end{itemize}

Таким образом, итоговая архитектура должна обеспечивать гибкость, производительность и 
простоту использования, позволяя пользователям не только взаимодействовать с виртуальными 
аватарами, но и активно участвовать в процессе их создания и улучшения.