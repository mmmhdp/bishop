\cchapter{ВВЕДЕНИЕ}                         
\addcontentsline{toc}{chapter}{ВВЕДЕНИЕ}

С ростом интереса к персонализированным цифровым технологиям моделирование речи, поведения  
и стиля мышления конкретных людей приобретает всё большую значимость. Цифровые аватары,  
способные воспроизводить облик, голос и характерную манеру общения конкретной личности,  
становятся частью инфраструктуры новых медиаформатов и пользовательского взаимодействия.  
В условиях стремительного развития генеративных моделей и широкого внедрения искусственного  
интеллекта в повседневную коммуникацию технологии создания интеллектуальных двойников  
находят применение в самых разных сферах — от сохранения цифровой памяти до коммерческих  
виртуальных ассистентов и инфлюенсеров, не привязанных к физическому носителю.

Особую актуальность такие решения приобретают в контексте работы с публичными, культурными  
или научными фигурами. Возможность сохранить их речевую манеру и поведенческую логику в виде  
доступного цифрового интерфейса позволяет не только продлить культурное присутствие личности,  
но и сделать его интерактивным и образовательным. Кроме того, в условиях растущего запроса на  
персонализированные сервисы и форматы «виртуального присутствия», технология создания  
интеллектуальных аватаров может найти применение в разработке цифровых консультантов,  
обучающих систем, а также в области цифрового наследия — как инструмент архивирования опыта,  
взглядов и коммуникативной стилистики конкретного человека.

Цель данной выпускной квалификационной работы — разработка сервиса под названием «Bishop»,  
предназначенного для общения пользователей с цифровыми двойниками или «аватарами» реальных  
личностей. Под «аватаром» в данном контексте понимается интеллектуальная компьютерная модель,  
способная воспроизводить стиль речи, манеру ведения беседы и эмоциональные реакции конкретного  
человека.

Идея работы заключается в том, чтобы предоставить пользователю не просто «чат-бота»,  
а максимально близкого к реальному собеседника. При этом формат взаимодействия выходит за рамки  
текстовых сообщений. Сервис, помимо текстового чата, обеспечивает генерацию аудиоответов,  
что создаёт иллюзию живой речи. Такая функциональность востребована в самых разных сценариях:  
от сохранения культурного и научного наследия выдающихся лекторов и артистов до интерактивных  
образовательных платформ, где диалог с «виртуальным преподавателем» может способствовать более  
глубокому усвоению материала.

Важной отличительной чертой сервиса является возможность обучения аватара на широком спектре  
материалов. Речь идёт не только об изначальном корпусе данных (тексты, аудио и видео с участием  
реального прототипа), но и о регулярном пополнении этих данных через пользовательский интерфейс.  
Например, в учебных целях можно загрузить дополнительную лекцию или интервью, что расширит  
«знания» цифрового двойника и углубит способность имитировать манеру общения конкретной  
личности. В результате пользователь получает впечатление общения с «живым» человеком,  
обладающим определённой индивидуальностью и способным рассуждать на разные темы  
в присущей ему манере.
