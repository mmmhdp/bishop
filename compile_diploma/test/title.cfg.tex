%%%%%%%%%%%%%%%%%%%%%%%%%%%%%%%%%%%%%%%%%%%%%%%%%%%%%%%%%%%%%%%%%%%%
% Информация для заполнения титульного листа и задания на ВКР

% ФИО студента
\AuthorShortName{Мальцев Н.А. Макеев А.А.  Исаков Я.И.}
\AuthorFullName{Мальцеву Никите Алексеевичу, Макееву Ацамазу Aлановичу, Исакову Ярославу Ильичу} % полностью в родительном падеже
% Номер группы
\AuthorGroup{БИВТ-21-16 и БИВТ-21-17}

% Тема работы
\ThesisTitle{Разработка сервиса создания и тестирования ИИ-аватаров}

% Научный руководитель
\SupervisorShort{Агабубаев Аслан Такабудинович}
\SupervisorFull{Агабубаев Аслан Такабудинович}
\SupervisorDegree{Старший преподаватель кафедры АСУ}

% Нормоконтроль
\GOSTComplianceReviewer{Агабубаев А.Т.}
% Проверка на заимствования
\PlagiarismChecker{Чумакова М.Ю.}

% Название кафедры, ФИО зав. кафедрой
\DeptShort{АСУ}
\DeptFull{АВТОМАТИЗИРОВАННЫХ СИСТЕМ УПРАВЛЕНИЯ}
\DeptHead{Темкин И.О.}

% Название института, ФИО директора
\InstShort{ИТКН}
\InstFull{ИНФОРМАЦИОННЫХ ТЕХНОЛОГИЙ И КОМПЬЮТЕРНЫХ НАУК}
\InstHead{Солодов С.В.}

% Шифр специальности, краткое и полное наименование специальности
\SpecShort{09.03.01 ИВТ}
\SpecFull{09.03.01 ИНФОРМАТИКА И ВЫЧИСЛИТЕЛЬНАЯ ТЕХНИКА}

% Место и год выполнения работы
\City{Москва}
\Year{2025}

% Цель работы
\ThesisPurpose{Реализовать систему интерактивного общения с обучаемыми цифровыми аватарами, позволяющую генерировать персонализированные текстовые и аудиоответы и обучать аватаров на пользовательских материалах}

% Исходные данные
\ThesisData{для TTS модели - Russian Open Speech To Text датасет, основой для LLM модели выстуают данные, поступающие в систему от пользователя}

% Основная литература, в том числе:
% Монографии, учебники и т.\,п.
\ThesisBooks{%
~Richardson~C. Microservices Patterns: With examples in Java.~--- Manning Publications, 2018.~--- 520~p.;
~Kleppmann~M. Designing Data-Intensive Applications: The Big Ideas Behind Reliable, Scalable, and Maintainable Systems.~--- O'Reilly Media, 2017.~--- 616~p.}

% Отчеты по НИР, диссертации, дипломные работы и т.\,п.
\ThesisReports{%
~Jemine~C. Real-Time Voice Cloning: Unpublished master's thesis.~--- Université de Liège, Liège, Belgique, 2019.
}

% Периодическая литература (журналы)
\ThesisJournals{%
~Vaswani~A., Shazeer~N., Parmar~N. et al. Attention Is All You Need //~CoRR.~--- 2017.~--- abs/1706.03762;
~Ye~Z., Zhu~X., Chan~C.-M. et al. Llasa: Scaling Train-Time and Inference-Time Compute for Llama-based Speech Synthesis //~arXiv preprint arXiv:2502.04128.~--- 2025;
~Khalid~Z., Li~K., Sah~M. et al. Transformers and audio detection tasks: An overview //~Digital Signal Processing.~--- 2025.~--- V.~158.~--- Article 104956;
~Touvron~H., Lavril~T., Izacard~G. et al. LLaMA: Open and Efficient Foundation Language Models //~arXiv preprint arXiv:2302.13971.~--- 2023;
~Radford~A., Narasimhan~K., Salimans~T., Sutskever~I. Improving Language Understanding by Generative Pre-Training //~OpenAI.~--- 2018;
~Mistral AI. Mistral NeMo: A 12B Parameter Language Model with 128k Context Length //~Mistral AI Blog.~--- 2024.~--- URL;
~Nanosemantics. «Наносемантика» разработала для ЛДПР первый в мире политический алгоритм — нейросеть «Жириновский» //~Nanosemantics Blog.~--- 2023.}

% Справочники и методическая литература (в том числе литература 
% по методам обработки экспериментальных данных)
\ThesisManuals{%
~Ramírez~S. FastAPI: High-performance web framework for building APIs with Python.~--- 2018;
~Howard~J. FastHTML: The fastest way to create an HTML app.~--- 2024;
~Apache Software Foundation. Apache Kafka;
~Docker Inc. Docker: Accelerated Container Application Development.~--- 2023;
~Wolf~T., Debut~L., Sanh~V. et al. Transformers: State-of-the-Art Natural Language Processing //~EMNLP System Demonstrations.~--- 2020.~--- P.~38--45;
~Gölge~E. et al. Coqui TTS: A deep learning toolkit for Text-to-Speech.~--- v1.4.~--- 2021;
~MetaVoice Team. MetaVoice-1B: Foundational Model for Human-like, Expressive TTS.~--- 2024.}


% Основная литература, в том числе:
% Монографии, учебники и т.\,п.
%\ThesisBooks{}

% Отчеты по НИР, диссертации, дипломные работы и т.\,п.
%\ThesisReports{}

% Периодическая литература (журналы)
%\ThesisJournals{}

% Справочники и методическая литература (в том числе литература 
% по методам обработки экспериментальных данных)
%\ThesisManuals{}

% Перечень основных этапов исследования и форма промежуточной 
% отчётности по каждому этапу
\ThesisStages{%
литературный обзор и анализ предметной области --- письменный отчёт;
анализ требований к системе --- письменный отчёт;
проектирование концептуальной и логической модели системы --- письменный отчёт;
проектирование архитектуры на основе требований --- письменный отчёт;
проектирование интерфейса системы --- письменный отчёт;
разработка и тестирование системы --- письменный отчёт.}

% Аппаратура и методики, которые должны быть использованы в работе
\ThesisEquipment{
Аппаратура --- 
персональный компьютер (Ubuntu~24.04.2 LTS, ядро Linux~6.11.0-25-generic; процессор Intel Core i7-7500U @ 2.70–3.50\,ГГц (2 ядра, 4 потока); ОЗУ 15 Гбайт; Swap 4 Гбайт);
методики --- 
ГОСТ Р 51904-2011 Программное обеспечение встроенных систем. 
Общие требования к разработке и документированию,
ГОСТ Р ИСО/МЭК 20741-2019 Системная и программная инженерия. 
Руководство для оценки и выбора инструментальных средств программной инженерии, 
ГОСТ Р 57100-2016/ISO/IEC/IEEE 42010:2011 
Системная и программная инженерия. Описание архитектуры, 
ГОСТ Р 56920-2016/ISO/IEC/IEEE 29119:2013 
Системная и программная инженерия. Тестирование программного обеспечения.
}

% Использование ЭВМ
\ThesisComputer{
среда разработки Neovim + tmux;
язык программирования Python 3.10;
менеджер зависимостей uv;
веб-фреймворк FastAPI;
клиентская библиотека HTMX + FastHTML;
контейнеризация Docker + Docker Compose;
брокер сообщений Apache Kafka;
СУБД PostgreSQL;
объектное хранилище MinIO (S3-совместимое);
key-value база Redis;
библиотеки ML HuggingFace Transformers, Torchaudio;
система компьютерной вёрстки \LaTeX;
система контроля версий Git;
обучение и прототипирование моделей в средах Google Colab и Kaggle;
}

% Перечень (примерный) основных вопросов, которые должны быть 
% рассмотрены и проанализированы в литературном обзоре
\ThesisLitReview{
персонализация и адаптация NLP-моделей;
технологии синтеза речи и voice cloning;
архитектуры микросервисов для ML-сервисо;
методы извлечения текста из аудио/видео;
UX/UI для диалоговых систем;
подходы к постоянному дообучению моделей на пользовательских данных;
организация масштабируемого хранения мультимедиа;
}

% Перечень (примерный) графического и иллюстрированного материала
\IllustrMaterials{
график динамики патентной активности в области цифровых аватаров;
архитектурная C4-диаграмма сервиса;
UML-диаграммы (кейсов, классов, последовательностей);
BPMN-диаграмма бизнес-процесса обучения и взаимодействия;
диаграмма пользовательского потока (user flow);
макеты и скриншоты интерфейса веб-приложения;
диаграммы выполнения микросервисов и обмена сообщениями;
}

% Дата выдачи задания
\DateAssignment{<<19>> декабря 2024\,г.}

% Дата утверждения задания (должна быть не раньше даты выдачи задания)
\DateApproval{<<19>> декабря 2024\,г.}
