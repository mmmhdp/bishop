\chapter{Результаты}

Раздел посвящён краткому изложению основных результатов
экспериментального тестирования системы. Каждый этап
подтверждён логами и сопровождается иллюстрациями,
представленными в приложении~\ref{app:B}.

\section{Инициализация системы}

После запуска всех микросервисов система переходит в состояние
полной готовности. Доступность компонентов подтверждается с
помощью механизма периодического \textit{health-check},
реализованного через брокер Kafka. Диагностические сообщения
успешно обрабатываются всеми службами (см.
рис.~\ref{fig:res-healt-check}).

\section{Регистрация и создание аватара}

Пользователь успешно проходит регистрацию и авторизацию,
после чего инициализируется пустой список аватаров.
Создание нового аватара выполняется без ошибок, и он
получает статус доступности (см.
рис.~\ref{fig:res-signup-login-frontend},
\ref{fig:res-signup-login-backend},
\ref{fig:res-bk-create-avatar}).

\section{Загрузка обучающих материалов}

Текстовые и аудиофайлы проходят валидацию и сохраняются
в объектное хранилище. Метаданные регистрируются и становятся
доступными для дальнейшего использования (см.
рис.~\ref{fig:res-fr-upload-materials},
\ref{fig:res-bk-upload-materials}).

\section{Обучение аватара}

Процедура дообучения запускается после получения команды
\texttt{train\_start}. Компоненты обрабатывают данные, а
по завершении аватар возвращается в состояние
\texttt{available} (см. рис.~\ref{fig:res-bk-start-train},
\ref{fig:res-llm-start-train},
\ref{fig:res-bk-stop-train},
\ref{fig:res-llm-stop-train}).

\section{Диалог и синтез речи}

Пользователь отправляет сообщение. Система формирует
текстовый ответ, выполняет синтез речи и возвращает
результат в интерфейс (см. рис.~\ref{fig:res-bk-start-gen-llm},
\ref{fig:res-llm-inference},
\ref{fig:res-bk-middle-gen-llm-done},
\ref{fig:res-sound-inference},
\ref{fig:res-bk-end-inference}).
