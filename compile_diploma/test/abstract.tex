\pdfbookmark{АННОТАЦИЯ}{Referat}
\cchapter{АННОТАЦИЯ}                       % Заголовок

% Число рисунков, таблиц, источников, приложений и общее 
% количество страниц ВКР подсчитываются автоматически.

Выпускная квалификационная работа изложена на  
\formbytotal{TotPages}{страниц}{е}{ах}{ах},
содержит
\formbytotal{totalcount@figure}{рисун}{ок}{ка}{ков},
\formbytotal{totalcount@table}{таблиц}{у}{ы}{},
\formbytotal{citenum}{источник}{}{а}{ов}, 
\formbytotal{totalappendix}{приложени}{е}{я}{й}.
\bigskip

\noindent
ЦИФРОВОЙ АВАТАР, ГЕНЕРАЦИЯ ТЕКСТА, СИНТЕЗ РЕЧИ, ДИАЛОГОВАЯ СИСТЕМА
\bigskip

%Перечень ключевых слов должен характеризовать содержание реферируемой 
%ВКР. Он должен включать до пяти ключевых слов в~именительном падеже, 
%напечатанных последовательно через запятые. 
%
%Текст реферата, помимо сведений об объёме ВКР и ключевых слов, 
%включает: сущность выполненной работы (её цель, объект исследования), 
%описание методов исследования и аппаратуры; конкретные сведения, 
%раскрывающие содержание основной части ВКР; краткие выводы 
%об особенностях работы, её эффективности, возможности и области 
%применения полученных результатов, их новизну. Каждая фраза реферата 
%должна быть носителем информации. Реферат не должен подменять 
%оглавления и должен быть достаточно полным. Объём реферата "--- 
%не более одной страницы.

Данная выпускная квалификационная работа посвящена разработке и экспериментальной апробации веб-сервиса, ориентированного на создание и персонализацию виртуальных собеседников — цифровых аватаров, способных поддерживать диалог в текстовой форме с последующим синтезом речи. Целью исследования являлось проектирование архитектурного и программного решения, обеспечивающего возможность загрузки пользовательских материалов для модификации поведения аватара и адаптации языковой модели под индивидуальные особенности коммуникации.

В качестве метода построения системы использован модульный подход: её компоненты разделены на отдельные подсистемы, отвечающие за обработку пользовательских данных, хранение структурированной и мультимедийной информации, а также организацию обмена сообщениями между элементами системы. Пользовательский интерфейс спроектирован с акцентом на интерактивность, что обеспечивает низкое время отклика и удобство при взаимодействии с цифровым собеседником.

Генерация текстовых реплик осуществляется с применением русскоязычной языковой модели, подстраиваемой под стилистические особенности конкретного пользователя. Синтез речи реализован с использованием заранее обученной голосовой модели, в которую на этапе генерации передаются пользовательские аудиофрагменты для имитации индивидуальных голосовых характеристик. Таким образом, система позволяет воспроизводить реплики, близкие по тембру и интонации к оригинальному голосу пользователя.

Практическая значимость разработки заключается в построении минимально жизнеспособного прототипа, демонстрирующего возможность создания персонализированных цифровых аватаров с текстовым и голосовым взаимодействием. Основной сервис был развёрнут в локальной среде и включает расширенный функционал для пользовательской настройки поведения аватара. Отдельные компоненты, такие как прототип генерации речи, разрабатывались с использованием облачных вычислительных платформ. Полученные результаты могут быть применимы в сферах образования, цифровой гуманитаристики и интерактивных мультимедийных приложений. Новизна работы состоит в объединении средств кастомизации как речевого, так и языкового поведения в рамках единого пользовательского интерфейса.



\clearpage
\pdfbookmark{ABSTRACT}{Abstract}
\cchapter{ABSTRACT}                       % Заголовок


The Bachelor's thesis has
\formbytotalen{TotPages}{page}{}{s},
\formbytotalen{totalcount@figure}{figure}{}{s},
\formbytotalen{totalcount@table}{table}{}{s},
\formbytotalen{citenum}{reference}{}{s}, 
\formbytotalen{totalappendix}{appendi}{x}{cies}.
\bigskip

\noindent
DIGITAL AVATAR, TEXT GENERATION, SPEECH SYNTHESIS, CHAT SYSTEM
\bigskip

%As any dedicated reader can clearly see, the Ideal of
%practical reason is a representation of, as far as I know, the things
%in themselves; as I have shown elsewhere, the phenomena should only be
%used as a canon for our understanding. The paralogisms of practical
%reason are what first give rise to the architectonic of practical
%reason. As will easily be shown in the next section, reason would
%thereby be made to contradict, in view of these considerations, the
%Ideal of practical reason, yet the manifold depends on the phenomena.
%
%Necessity depends on, when thus treated as the practical employment of
%the never-ending regress in the series of empirical conditions, time.
%Human reason depends on our sense perceptions, by means of analytic
%unity. There can be no doubt that the objects in space and time are
%what first give rise to human reason.

This bachelor's thesis is devoted to the development and experimental evaluation of a web-based service aimed at the creation and personalization of virtual interlocutors—digital avatars capable of conducting text-based dialogues with subsequent speech synthesis. The goal of the research was to design an architectural and software solution that enables users to upload their own materials in order to modify avatar behavior and adapt a language model to individual communication styles.

A modular approach was used in building the system: its components are organized into distinct subsystems responsible for processing user data, storing structured and multimedia content, and coordinating message exchange between elements. The user interface is designed with a focus on interactivity, ensuring low response time and convenient engagement with the digital interlocutor.

Text generation is carried out using a Russian-language language model that can be adapted to the stylistic features of a specific user. Speech synthesis is implemented using a pre-trained voice model, which, during the generation stage, receives user audio samples to imitate individual vocal characteristics. As a result, the system can produce speech that closely resembles the user’s original tone and intonation.

The practical significance of the project lies in the development of a minimum viable prototype that demonstrates the feasibility of creating personalized digital avatars capable of both textual and vocal interaction. The main service was deployed in a local environment and includes extended functionality for customizing avatar behavior. Certain components, such as the speech generation prototype, were developed using cloud-based computing platforms. The results obtained may find application in the fields of education, digital humanities, and interactive multimedia systems. The novelty of this work lies in the integration of both speech and language personalization tools within a unified user interface.
