\chapter{Оформление различных элементов}\label{ch:ch1}

\section{Форматирование текста}\label{sec:ch1/sec1}

Предложение может содержать \textbf{полужирный текст} и \textit{курсив}. 
Сочетать \textit{\textbf{полужирное начертание и~курсив}}
крайне не рекомендуется.
Запрещается использовать \uline{подчёркивание}. 


\section{Списки}\label{sec:ch1/sec2}

\noindent Нумерованный список:
\begin{enumerate}
    \item Первый пункт.
    \item Второй пункт.
    \item Третий пункт.
\end{enumerate}

\noindent Пример маркированного списка:
\begin{itemize}
    \item пункт первый;
    \item пункт второй;
    \item пункт третий.
\end{itemize}

\noindent Вложенные списки:
\begin{itemize}
    \item Имеется маркированный список.
          \begin{enumerate}
              \item В нём лежит нумерованный список,
              \item в котором
                    \begin{itemize}
                        \item лежит ещё один маркированный список.
                    \end{itemize}
          \end{enumerate}
\end{itemize}

\noindent Нумерованный вложенный список:
\begin{enumerate}
    \item пункт первый;
    \item пункт второй:
    \begin{enumerate}
            \item Учитывая ключевые сценарии поведения, 
            высокотехнологичная концепция общественного уклада играет 
            определяющее значение для соответствующих условий 
            активизации:
            \begin{enumerate}
                \item третий;
                \item уровень;
                \item вложенности.
            \end{enumerate}
            \item второй подпункт;
        \end{enumerate}
    \item пункт третий:
        \begin{enumerate}
            \item второй;
            \item уровень;
            \item вложенности.
        \end{enumerate}
\end{enumerate}


\section{Ссылки на литературу}\label{sec:ch1/sec3}

Ниже приведены примеры ссылок на 
\begin{itemize}
  \item книги: \cite{Sychev,Sokolov,Gaidaenko},
  \item многотомные издания: \cite{Lermontov,Russell},

  \item статьи: \cite{Berestova,Janiesch},
  \item труды конференций: \cite{Conference},
  \item препринты: \cite{arXiv,arXiv2},

  \item электронные ресурсы: \cite{Encyclopedia,Nasirova},
  \item диссертации: \cite{Lagkueva,Pokrovski,Belozerov},
  \item авторефераты диссертаций: \cite{Sirotko,Lukina},
  \item патенты: \cite{patent2,patent3,patbib1},
  \item свидетельство о регистрации программы для ЭВМ: \cite{progbib1},
  \item нормативные правовые акты: \cite{Constitution,FamilyCode},
  \item стандарты: \cite{Gost.7.0.53,standard3},
  \item депонированные научные работы: \cite{Razumovski},
  \item отчёты о научно-исследовательской работе: \cite{Methodology}.
\end{itemize}

Ссылки с использованием дополнительных опций biblatex: \cite{sobenin_kdv,initials}.

Ссылки с указанием страниц: \cite[с.~54]{Sokolov}\cite[с.~36]{Gaidaenko}.


\section{Ссылки на формулы, таблицы, рисунки и~т.\,п.}\label{sec:ch1/sec4}

Ссылки на приложения: Приложение~\ref{app:A}, Приложение~\ref{app:C1}.

Ссылка на формулу: формула~\eqref{eq:equation1}.

Ссылка на изображение: рисунок~\ref{fig:2}.

Общепринятым является добавление к ссылкам префикса, характеризующего 
тип элемента. Это не является строгим требованием, но~позволяет лучше 
ориентироваться в документах большого размера.
Например, для ссылок на~рисунки используется префикс \texttt{fig},
для ссылки на~таблицу "--- \texttt{tab}.
Например, \verb+\label{fig:example}+; \verb+\ref{tab:test1}+; \verb+label={lst:external1}+.
В~таблице \ref{tab:stand_pref} приведены стандартные префиксы для 
различных типов ссылок.

\begin{table}[htbp]
    \centering
    \caption{Стандартные префиксы ссылок}
    \label{tab:stand_pref}
    \begin{tblr}{colspec={ll},
                row{1} = {font=\bfseries},
                hline{2}={1}{-}{},
                hline{2}={2}{-}{},
                hline{1,Z}={solid},
                vlines,vline{1,3}={abovepos=1,belowpos=1},
                }
        Элемент           & Префикс \\
        Глава             & ch:     \\
        Секция            & sec:    \\
        Подсекция         & subsec: \\
        Рисунок           & fig:    \\
        Таблица           & tab:    \\
        Уравнение         & eq:     \\
        Листинг программы & lst:    \\
        Элемент списка    & itm:    \\
        Алгоритм          & alg:    \\
        Секция приложения & app:    \\
    \end{tblr}
\end{table}

Для упорядочивания ссылок можно использовать разделительные символы.
Например, \verb+\label{fig:schemes/my_scheme}+ 
или \\ \verb+\label{lst:dts/linked_list}+.




\section{Набор формул}\label{sec:ch1/sec5}

\subsection{Ненумерованные одиночные формулы}\label{subsec:ch1/sec5/sub1}

Для добавления формул можно использовать пары \verb+$+\dots\verb+$+ 
и \verb+$$+\dots\verb+$$+, но~они считаются устаревшими.
Лучше использовать их функциональные аналоги \verb+\(+\dots\verb+\)+ 
и \verb+\[+\dots\verb+\]+.

Пример внутритекстовой формулы \(x \approx \sin x\), то есть формулы,
которая располагается непосредственно в~тексте.

Пример выключной формулы:
\[
    (x_1+x_2)^2 = x_1^2 + 2 x_1 x_2 + x_2^2
\]

Согласно русской типографской традиции в стилевом файле используется
прямое начертание греческих букв:
\[
    \alpha\beta\gamma\delta\epsilon\epsilon\zeta\eta\theta%
    \vartheta\iota\kappa\varkappa\lambda\mu\nu\xi\pi\varpi\rho\varrho%
    \sigma\varsigma\tau\upsilon\phi\phi\chi\psi\omega\Gamma\Delta%
    \Theta\Lambda\Xi\Pi\Sigma\Upsilon\Phi\Psi\Omega
\]
\[
    \boldsymbol{\alpha\beta\gamma\delta\epsilon\epsilon\zeta\eta%
        \theta\vartheta\iota\kappa\varkappa\lambda\mu\nu\xi\pi\varpi\rho%
        \varrho\sigma\varsigma\tau\upsilon\phi\phi\chi\psi\omega\Gamma%
        \Delta\Theta\Lambda\Xi\Pi\Sigma\Upsilon\Phi\Psi\Omega}
\]

В формулах можно использовать разные математические алфавиты:
\begin{align*}
    \mathcal{ABCDEFGHIJKLMNOPQRSTUVWXYZ} \\
    \mathfrak{ABCDEFGHIJKLMNOPQRSTUVWXYZ} \\
    \mathbb{ABCDEFGHIJKLMNOPQRSTUVWXYZ}
\end{align*}



\subsection{Многострочные формулы}\label{sub:ch1/sec5/sub2}

Вот так можно написать несколько формул, которые выровнены по
знаку равенства:
\begin{align*}
    f(E,T) & = \frac{1}{e^{\frac{E - E_F}{kT}} - 1}, \\
    f(E,T) & = \frac{1}{e^{\frac{E - E_F}{kT}} + 1}.
\end{align*}

Выровнять систему ещё и по переменной \( x \) можно, используя окружение
\verb|alignedat| из пакета \verb|amsmath|. Вот так:
\[
|x| = \left\{
        \begin{alignedat}{2}
            & &x, \quad & \text{eсли } x \geq 0 \\
            &-&x, \quad & \text{eсли } x < 0
        \end{alignedat}
    \right.
\]
Здесь первый амперсанд (в исходном \LaTeX\ описании формулы) означает
выравнивание по~левому краю, второй "--- по~\( x \), а~третий "--- по~слову
<<если>>. Команда \verb|\quad| делает большой горизонтальный пробел.


Для вёрстки простых матриц удобно окружение \texttt{pmatrix}:
\[
\begin{pmatrix}
    a_{11} & \ldots & a_{1n} \\
    \vdots & \ddots & \vdots \\
    a_{n1} & \ldots & a_{nn}
\end{pmatrix}
\]

Для более сложных матриц лучше использовать пакет \texttt{nicematrix}.
Ниже приведено несколько примеров, взятых из документации к этому пакету.
\[
\begin{pNiceMatrix}[first-row,first-col,columns-width=7mm,
                    code-for-first-col=\scriptstyle\color{red},
                    code-for-first-row=\scriptstyle\color{blue}]
       & C_1    & C_2    & \Cdots & C_n    \\
L_1    & a_{11} & a_{12} & \Cdots & a_{1n} \\
L_2    & a_{21} & a_{22} & \Cdots & a_{2n} \\
\Vdots & \Vdots & \Vdots & \Ddots & \Vdots \\
L_n    & a_{n1} & a_{n2} & \Cdots & a_{nn} \\
\end{pNiceMatrix}
\]

\[
B = \begin{bNiceMatrix}[margin,hvlines,columns-width=5mm]
\Block{3-3}<\Large>{A} & & & 0 \\
& \hspace*{1cm} & & \Vdots \\
& & & 0 \\
0 & \Cdots& 0 & 0
\end{bNiceMatrix}
\]

\[
\begin{pNiceArray}{*6c|c}[columns-width=auto,last-col,code-for-last-col=\scriptstyle]
1 & 1 & 1 &\Cdots & & 1 & 0 & \\
0 & 1 & 0 &\Cdots & & 0 & & L_2 \gets L_2-L_1 \\
0 & 0 & 1 &\Ddots & & \Vdots & & L_3 \gets L_3-L_1 \\
& & &\Ddots & & & \Vdots & \Vdots \\
\Vdots & & &\Ddots & & 0 & \\
0 & & &\Cdots & 0 & 1 & 0 & L_n \gets L_n-L_1
\end{pNiceArray}
\]




\subsection{Нумерованные формулы}\label{subsec:ch1/sec5/sub3}

Пример нумерованной формулы:
\begin{equation}\label{eq:equation1}
    e = \lim_{n \to \infty} \left( 1 + \frac{1}{n} \right)^n
\end{equation}

Ссылка на формулу~\eqref{eq:equation1}.

Для нумерованных формул \verb|aligned| делает вертикальное
выравнивание номера формулы по центру формулы:
\begin{equation}\label{eq:2}
    \begin{aligned}
        j &= - D \nabla c, \\
        \frac{\partial c}{\partial t} &= \nabla( D \nabla c)
    \end{aligned}
\end{equation}

Для длинной формулы, которая не помещается на одной строке можно использовать
окружение \verb|multline|.
\begin{multline}\label{eq:equation31}
    1 + 2 + 3 + 4 + 5 + 6 + 7 + \dots + \\
    + 50 + 51 + 52 + 53 + 54 + 55 + 56 + 57 + \dots + \\
    + 96 + 97 + 98 + 99 + 100 = 5050
\end{multline}

Сделать так, чтобы номер формулы стоял напротив средней строки, можно,
используя окружение \verb|multlined| вместо \verb|multline| 
внутри окружения \verb|equation|. Результат получается вот такой:
\begin{equation}\label{eq:equation3}
    \begin{multlined}
        1 + 2 + 3 + 4 + 5 + 6 + 7 + \dots + \\
        + 50 + 51 + 52 + 53 + 54 + 55 + 56 + 57 + \dots + \\
        + 96 + 97 + 98 + 99 + 100 = 5050
    \end{multlined}
\end{equation}

Уравнение~\eqref{eq:subeq_1} демонстрирует возможности
окружения \verb|\subequations|.
\begin{subequations}
    \label{eq:subeq_1}
    \begin{gather}
        y = x^2 + 1 \label{eq:subeq_1-1} \\
        y = 2 x^2 - x + 1 \label{eq:subeq_1-2}
    \end{gather}
\end{subequations}

Ссылки на отдельные уравнения~\eqref{eq:subeq_1-1} и~\eqref{eq:subeq_1-2}.

Пример формулы с пояснением значений символов. Их приводят 
непосредственно после формулы, на следующей строке, в той же 
последовательности, в~которой они присутствуют в~формуле:
\[
\rho = \frac{m}{V},
\]
\where{%
    $\rho$ "--- плотность образца, $\text{г}/\text{см}^3$; \\
    $m$ "--- масса образца, г; \\
    $V$ "--- объём образца, $\text{см}^3$.}

Ещё один пример формулы с~пояснением. 
Статистическая мера $\operatorname{tf-idf}$ рассчитывается 
следующим образом:
\begin{equation}\label{eq:tf-idf}
\operatorname{tf-idf}(t, D) = \operatorname{tf}(t, D) \cdot \operatorname{idf}(t),
\end{equation}
\where{%
    $t$ "---  термин из поискового запроса; \\
    $D$ "--- документ; \\
    $\operatorname{tf}(t, D)$ "--- отношение числа вхождений термина $t$ к общему числу слов документа~$D$; \\
    $\operatorname{idf}(t)$ "--- инверсия частоты, с которой термин $t$ встречается в документах коллекции.
}





\subsection{Заголовки с формулами: \texorpdfstring{\(a^2 + b^2 = c^2\)}{%
        a\texttwosuperior\ + b\texttwosuperior\ = c\texttwosuperior},
    \texorpdfstring{\(\left\vert\textrm{{Im}}\Sigma\left(
            \protect\varepsilon\right)\right\vert\approx const\)}{|ImΣ (ε)| ≈ const},
    \texorpdfstring{\(\sigma_{xx}^{(1)}\)}{σ\_\{xx\}\textasciicircum\{(1)\}}
}\label{sub:with_math}

Пакет \texttt{hyperref} берёт текст для закладок в pdf-файле из~аргументов
команд типа \verb|\section|, которые могут содержать математические формулы,
а~также изменения цвета текста или шрифта, которые не отображаются в~закладках.
Чтобы использование формул в заголовках не вызывало в~логе компиляции появление
предупреждений типа <<\texttt{Token not allowed in~a~PDF string
    (Unicode):(hyperref) removing...}>>, следует использовать конструкцию
\verb|\texorpdfstring{}{}|, где в~первых фигурных скобках указывается
формула, а~во~вторых "--- запись формулы для закладок.




% Данной командой рекомендуется завершать раздел чтобы плавающие объекты
% (рисунки/таблицы) размещались в конце текущего раздела и не 
% переносились в начало следующего.
\FloatBarrier
